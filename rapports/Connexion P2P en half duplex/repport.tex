\documentclass{article}
\usepackage[utf8]{inputenc}
\usepackage[T1]{fontenc}
\usepackage[french]{babel}
\usepackage{listings}
\usepackage{xcolor}
\usepackage{geometry}
\geometry{margin=2cm}

\definecolor{codegreen}{rgb}{0,0.6,0}
\definecolor{codegray}{rgb}{0.5,0.5,0.5}
\definecolor{codepurple}{rgb}{0.58,0,0.82}
\definecolor{backcolour}{rgb}{0.95,0.95,0.92}

\lstdefinestyle{mystyle}{
    backgroundcolor=\color{backcolour},   
    commentstyle=\color{codegreen},
    keywordstyle=\color{magenta},
    numberstyle=\tiny\color{codegray},
    stringstyle=\color{codepurple},
    basicstyle=\ttfamily\footnotesize,
    breakatwhitespace=false,         
    breaklines=true,                 
    captionpos=b,                    
    keepspaces=true,                 
    numbers=left,                    
    numbersep=5pt,                  
    showspaces=false,                
    showstringspaces=false,
    showtabs=false,                  
    tabsize=2
}

\lstset{style=mystyle}

\title{Système de Connexion Pair-à-Pair en Half Duplex}
\author{}
\date{}

\begin{document}

\maketitle

\section{Introduction}
Ce rapport présente le sous-système de connexion entre deux appareils développé en Python. Le système utilise une architecture client-serveur avec un mécanisme de codage personnalisé pour faciliter l'établissement de la connexion.

\section{Architecture Générale}
Le système se compose de deux composants principaux :
\begin{itemize}
    \item \textbf{server.py} : Application serveur qui écoute les connexions entrantes
    \item \textbf{client.py} : Application cliente qui initie la connexion au serveur
\end{itemize}

\section{Mécanisme de Connexion}

\subsection{Détection d'Adresse IP}
Le serveur utilise la fonction \texttt{get\_local\_ip()} pour déterminer automatiquement son adresse IP locale :
\begin{itemize}
    \item Priorité aux interfaces réseau (via \texttt{netifaces})
    \item Fallback sur l'adresse de l'hôte local
    \item Filtrage des adresses privées (192.168.*, 10.*)
\end{itemize}

\subsection{Système de Codage de Connexion}

\subsubsection{Base 64 Personnalisée}
Une base 64 personnalisée est définie : 

\texttt{"0123456789ABCDEFGHIJKLMNOPQRSTUVWXYZabcdefghijklmnopqrstuvwxyz!?"}

\subsubsection{Encodage (Côté Serveur)}
La fonction \texttt{encode\_connexion\_code()} transforme l'IP et le port en un code de 8 caractères :
\begin{enumerate}
    \item Conversion de l'IP en quatre segments binaires de 8 bits
    \item Conversion du port en binaire sur 16 bits
    \item Concatenation en une chaîne binaire de 48 bits
    \item Découpage en 8 segments de 6 bits
    \item Conversion de chaque segment en caractère base64
\end{enumerate}

\subsubsection{Décodage (Côté Client)}
La fonction \texttt{decode\_connexion\_code()} effectue l'opération inverse :
\begin{enumerate}
    \item Conversion de chaque caractère en valeur binaire 6 bits
    \item Reconstruction de la chaîne binaire complète (48 bits)
    \item Extraction des composants IP (32 bits) et port (16 bits)
    \item Conversion en format IP décimal et port entier
\end{enumerate}

\section{Protocole de Communication}

\subsection{Établissement de Connexion}
\begin{enumerate}
    \item Le serveur démarre et affiche le code de connexion
    \item Le client décode le code pour obtenir IP et port
    \item Connexion TCP standard via \texttt{socket.connect()}
    \item Message de bienvenue envoyé par le serveur
\end{enumerate}

\subsection{Système de Chat}
Le protocole de communication suit un modèle half-duplex :
\begin{itemize}
    \item Alternance stricte des messages client-serveur
    \item Taille de buffer fixe à 1024 octets
    \item Gestion basique de fin de session via intérruption clavier
\end{itemize}

\section{Avantages du Système}

\subsection{Simplicité d'Utilisation}
\begin{itemize}
    \item Code de connexion facile à communiquer (8 caractères)
    \item Détection automatique de l'adresse réseau
    \Interface texte simple et intuitive
\end{itemize}

\subsection{Robustesse}
\begin{itemize}
    \item Gestion des erreurs de connexion
    \item Fallback en cas d'échec de détection d'IP
    \item Fermeture propre des sockets
\end{itemize}

\subsection{Portabilité}
\begin{itemize}
    \item Utilisation de bibliothèques Python standard
    \item Compatible avec différentes plateformes
    \item Indépendant de la configuration réseau spécifique
\end{itemize}

\section{Limitations Identifiées}

\begin{itemize}
    \item Dépendance optionnelle à \texttt{netifaces} (installation requise)
    \item Gestion basique des erreurs réseau
    \item Code de connexion limité aux adresses IPv4
\end{itemize}

\end{document}